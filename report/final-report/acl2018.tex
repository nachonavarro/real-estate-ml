\documentclass[11pt,a4paper]{article}
\usepackage[hyperref]{acl2018}
\usepackage{times}
\usepackage{latexsym}

\usepackage{url}

\aclfinalcopy

\title{Using Geospatial Data to Find Investment Opportunities in the Real Estate Market}

\author{Chih Huang \\
  {\tt huangcs} \\
  {\tt @seas.upenn.edu} \\\And
  Ignacio Navarro \\
  {\tt inavarro} \\
  {\tt @seas.upenn.edu} \\\And
  Nikhil Ramesh \\
  {\tt nramesh} \\
  {\tt @seas.upenn.edu}
}

\date{\today}

\begin{document}
\maketitle
\begin{abstract}
  Valuating real estate properties is driven by a series of direct 
  factors (total square area, number of bedrooms, garage spaces, etc) 
  and indirect factors (crime rate, quality of school district, etc). However, private sellers tend to focus more on the former when setting 
  a listing price for their property. In particular, there are many
  geospatial features indirectly related to the property that sellers
  might easily overlook. Proximity to a church, a river, or a police
  station are all factors that play a role, albeit minor, in the valuation
  of a property. In this paper we determine by running a series of models
  if these geographical features do indeed play a decisive role in the valuation of a property. 
\end{abstract}

\section{Introduction}

One can use these results to find 
  investments opportunities by comparing the listing price on properties
  currently on the market with what the best model predicts the value 
  is and seeing if the property is really undervalued.

\subsection{Related Work}


\section{Data Acquisition and Exploration}

One can use these results to find 
  investments opportunities by comparing the listing price on properties
  currently on the market with what the best model predicts the value 
  is and seeing if the property is really undervalued.


Lorem ipsum dolor sit amet, consectetur adipisicing elit, sed do eiusmod
tempor incididunt ut labore et dolore magna aliqua. Ut enim ad minim veniam,
quis nostrud exercitation ullamco laboris nisi ut aliquip ex ea commodo
consequat. Duis aute irure dolor in reprehenderit in voluptate velit esse
cillum dolore eu fugiat nulla pariatur. Excepteur sint occaecat cupidatat non
proident, sunt in culpa qui officia deserunt mollit anim id est laborum.


\section{Models}

Lorem ipsum dolor sit amet, consectetur adipisicing elit, sed do eiusmod
tempor incididunt ut labore et dolore magna aliqua. Ut enim ad minim veniam,
quis nostrud exercitation ullamco laboris nisi ut aliquip ex ea commodo
consequat. Duis aute irure dolor in reprehenderit in voluptate velit esse
cillum dolore eu fugiat nulla pariatur. Excepteur sint occaecat cupidatat non
proident, sunt in culpa qui officia deserunt mollit anim id est laborum.

\section{Conclusion}

Lorem ipsum dolor sit amet, consectetur adipisicing elit, sed do eiusmod
tempor incididunt ut labore et dolore magna aliqua. Ut enim ad minim veniam,
quis nostrud exercitation ullamco laboris nisi ut aliquip ex ea commodo
consequat. Duis aute irure dolor in reprehenderit in voluptate velit esse
cillum dolore eu fugiat nulla pariatur. Excepteur sint occaecat cupidatat non
proident, sunt in culpa qui officia deserunt mollit anim id est laborum.

\subsection{Future Work}

\bibliographystyle{acl}
\bibliography{acl2018}

\end{document}
